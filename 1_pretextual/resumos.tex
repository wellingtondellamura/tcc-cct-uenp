% ---
% RESUMOS
% ---

% ---
% Resumo em Português (elemento obrigatório)
% ---
\begin{resumo}
 Segundo a \citeonline[3.1-3.2]{NBR6028:2003}, o resumo deve ressaltar o
 objetivo, o método, os resultados e as conclusões do documento. A ordem e a extensão
 destes itens dependem do tipo de resumo (informativo ou indicativo) e do
 tratamento que cada item recebe no documento original. O resumo deve ser
 precedido da referência do documento, com exceção do resumo inserido no
 próprio documento. (\ldots) As palavras-chave devem figurar logo abaixo do
 resumo, antecedidas da expressão Palavras-chave:, separadas entre si por
 ponto e finalizadas também por ponto.
\end{resumo}
% ---

% ---
% Resumo em Inglês (elemento obrigatório)
% ---
% O ambiente Abstract (com A maiúsculo) é definido no estilo dc-uel
\begin{Abstract}
 This is the english abstract. The Abstract in English should be faithful to the
 Resumo in Portuguese, but not a literal translation.
\end{Abstract}
% ---
